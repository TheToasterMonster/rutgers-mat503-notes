\chapter{Power Series}

\section{Last Time}
Recall that a function $f : \Omega \to \C$, where
$\Omega \subseteq \C$ is open, is holomorphic
at $z \in \Omega$ if
\[
  f'(z) := \lim_{h \to 0} \frac{f(z + h) - f(z)}{h}
.\]
exists. $f$ is holomorphic on $\Omega$ if it is
holomorphic at all $z \in \Omega$.

\begin{thm}{Theorem}
If a function $f = u + iv$ is holomorphic, then:
\begin{enumerate}[(i).]
  \item The complex derivative $f'(z)$ satisfies
    $f'(z) = \frac{\partial}{\partial z} f = \frac{\partial}{\partial x} f = u_x + iv_x$.
  \item The Jacobian $\left\lvert J \right\rvert = \left\lvert f'(z) \right\rvert^2 = u_x^2 + v_x^2$.
  \item (Cauchy-Riemann) $u_x = v_y$ and $u_y = -v_x$.
  \item The partial derivative with respect to $\overline{z}$ satisfies
    $\frac{\partial}{\partial \overline{z}} f = 0$.
\end{enumerate}
\end{thm}

\section{Practice with Holomorphic Functions}
\begin{mybox}{darkyellow}{Example 1}
  Let $f = (u, v) = u + iv$ where $u = x^2 + y^2$ and
  $v = 2xy$. Is $f$ holomorphic?
\end{mybox}

\begin{proof}[Solution]
  We can calculate
  \[u_y = 2y, \quad v_x = 2y.\]
  The Cauchy-Riemann equations are not satisfied since
  $u_y \ne v_x$, so $f$ is \textit{not} holomorphic.
\end{proof}

\begin{mybox}{darkyellow}{Example 2}
  Let $f = (u, v) = u + iv$ where $u = x^2 - y^2$ and
  $v = 2xy$. Is $f$ holomorphic?
\end{mybox}

\begin{proof}[Solution]
  Notice that $f$ is actually
  \[f(z) = z^2 = (x + iy)^2 = x^2 - y^2 + 2ixy.\]
  So $f$ \textit{is} holomorphic and it indeed satisfies
  the Cauchy-Riemann equations.
\end{proof}

\begin{examp}{Example 3}
  Let $f = (u, v) = u + iv$ where $u = x^2 - y^2$ and
  $v = -2xy$. Is $f$ holomorphic?
\end{examp}

\begin{proof}[Solution]
  Compute
  \[
    \frac{\partial}{\partial \overline{z}} f
    = \frac{1}{2}\left(\frac{\partial}{\partial x} (u + iv) - \frac{1}{i} \frac{\partial}{\partial y} (u + iv)\right)
    = 2x - 2yi = 2\overline{z}
  .\]
  So $f = {\overline{z}}^2$, which is not
  holomorphic. This is actually called
  \textit{anti-holomorphic}, when
  $\frac{\partial}{\partial z} f = 0$.
\end{proof}

\begin{examp}{Example 4}
  Let $f = (u, v) = u + iv$ where $u = x^2 + y^2$ and
  $v = 0$. Is $f$ holomorphic?
\end{examp}

\begin{proof}[Solution]
  Compute
  \[
  \frac{\partial}{\partial z} f =
  \frac{1}{2}\left(\frac{\partial}{\partial x}(u + iv) + \frac{1}{i} \frac{\partial}{\partial y} (u + iv)\right)
  = \frac{1}{2}\left(2x + \frac{1}{i} 2y\right)
  = x - iy = \overline{z}
  .\]
  Similarly we can find that
  $\frac{\partial}{\partial \overline{z}} = z$.
  Note that $f(z) = |z|^2 = z \overline{z}$, which
  is not holomorphic.
\end{proof}

\section{Converse of Cauchy-Riemann}
\begin{thm}{Theorem (Converse of Cauchy-Riemann equations)}
  If $u$ and $v$ are $\R^2$-differentiable near $z$ and
  satisfy the Cauchy-Riemann equations at $z$,
  then the complex function $f = u + iv$ is holomorphic
  at $z$.
\end{thm}

\begin{proof}
  For $h = h_1 + ih_2$, look at
  \[
    f(z + h) - f(z) = u(z + h) - u(z)
    + i(v(z + h) - v(z))
    = u_x h_1 + u_y h_2 + i(v_x h_1 + v_y h_2)
    + o(|h|)
  \]
  Then by the Cauchy-Riemann equations
  ($u_y \to -v_x$ and $v_y \to u_x$),
  \begin{align*}
    f(z + h) - f(z)
    &= u_x h_1 - v_x h_2 + i(v_x h_1 + u_x h_2) + o(|h|)
    = (u_x + iv_x) h_1 + i(u_x + iv_x) h_2 + o(|h|) \\
    &= (u_x + iv_x) (h_1 + ih_2) + o(|h|)
    = (u_x + iv_x) h + o(|h|)
  \end{align*}
  Dividing by $h$ gives
  \[
    \frac{f(z + h) - f(z)}{h} =
    \frac{\partial}{\partial x} f + o(1)
    = f'(z) + o(1)
  .\]
  So $f'(z)$ exists and $f$ is holomorphic at $z$.
\end{proof}

\begin{remark}
  It is not enough for $u_x, u_y, v_x, v_y$ to exist
  and satisfy the Cauchy-Riemann equations for
  holomorphicity. We really need $u$ and $v$ to be
  $\R^2$-differentiable. The classic counterexample here
  is
  \[
  f = \frac{z^5}{|z|^4}
  = \frac{z^5}{z^2 {\overline{z}}^2}
  = \frac{z^3}{{\overline{z}}^2}
  .\]
\end{remark}

\section{Analytic Functions}
Recall that $f : \Omega \to \C$ is analytic near
$z_0 \in \Omega$ if there exists $R > 0$ and a power series
representation
\[
  f(z) = \sum_{n \ge 0} a_n(z - z_0)^n
\]
which converges to $f$ absolutely for each $z$
in an open ball $B_R(z_0)$.

\begin{examp}{Example}
  \[\sum_{n \ge 0} \frac{z^n}{n!} = \exp(z), \quad \text{radius of convergence $R = \infty$.}\]
  \[
    \sum \frac{z^{2n}}{(2n)!} (-1)^n
    = \cos(z) = \frac{e^{iz} + e^{-iz}}{2}, \quad R = \infty.
  \]
  \[
    \sum \frac{z^{2n + 1}}{(2n + 1)!} (-1)^n
    = \sin(z) = \frac{e^{iz} - e^{-iz}}{2i}, \quad R = \infty.
  \]
  \[
    \sum z^n = \frac{1}{1 - z}, \quad \text{converges for $|z| < R = 1$}.
  \]
  \[
    \sum 2^n z^n
    = \frac{1}{1 - 2z}, \quad |z| < R = \frac{1}{2}
  .\]
  So even exponential growth in $a_n$ can result in a
  positive radius of convergence!
  \[
    \sum n! z^n, \quad \text{$R = 0$, i.e. it diverges everywhere}
  .\]
  This grows too fast. For $n > |z|$, $n! z^n \to \infty$,
  and $a_n \to 0$ is a necessary condition for convergence.
\end{examp}

So what happens in general?

\begin{thm}{Theorem (Hadamard)}
  Let $\sum a_n z^n$ be a power series. Then there exists
  some $0 \le R < \infty$ such that:
  \begin{enumerate}[(i)]
    \item For all $|z| < R$, the series converges absolutely.
    \item For all $|z| > R$, the series diverges.
  \end{enumerate}
  Furthermore, $R = \frac{1}{L}$ where
  $L = \limsup |a_n|^{1 / n}$.
\end{thm}

\begin{proof}[Proof of (i)]
  Assume $0 < R < \infty$ (the cases $R = 0$ and
  $R = \infty$ are left as exercises). Let
  $L = \limsup |a_n|^{1 / n}$ and let $|z| < R$.
  Note that $L = \limsup |a_n|^{1 / n}$ means that for
  all $\epsilon$, there exists $N$ such that
  for all $n > N$,
  \[
    |a_n|^{1 / n} < L + \epsilon
  .\]
  There exists $\epsilon$ such that
  $(L + \epsilon)|z| \le r < R = \frac{1}{L}$.
  So $L|z| < 1$. Let $\epsilon$ be small enough such
  that $(L + \epsilon)|z| \le r < 1$.
  Now look at the tail of the series:
  \[
    \sum_{n > N} |a_n| |z|^n
    \le \sum_{n > N} (L + \epsilon)^n |z|^n
    \le \sum_{n > N} r^n < \infty
  \]
  since $r < 1$. So the series converges absolutely
  for $|z| < R$.
  The proof for (ii) is left as an exercise.
\end{proof}

For example, if $a_n = n!$, recall that
$n! \approx n^n e^{-n} \sqrt{n}$ by Stirling's formula.
So
\[
  \left|\frac{1}{n!}\right|^{\frac{1}{n}}
  \approx \frac{e}{n} \to 0 = L
\]
and $R = 0$. So we should expect this to diverge
everywhere.

\begin{thm}{Theorem (Analytic implies holomorphic)}
  If $f = \sum a_n z^n$ has a radius of convergence
  $R > 0$, then $f$ is holomorphic on $|z| < R$.
  Moreover,
  \[
    f'(z) = \sum n a_n z^{n - 1}
  ,\]
  which has the same radius $R$ of absolute convergence.
\end{thm}

\begin{proof}[Sketch of proof]
  We know $f$ is analytic, so let
  \[S_N(z) = \sum_{n \le N} a_n z^n, \quad E_N(z) = \sum_{n > N} a_n z^n\]
  be the partial sums and error terms, respectively.
  Let
  \[
    g(z) = \sum n a_n z^{n - 1}
  .\]
  Note that $n^{1 / n} \to 1$, so clearly $g$ has
  absolute convergence for $|z| < R$. Then want to
  show that
  \[
    \left|\frac{f(z + h) - f(z)}{h} - g(z)\right| \to 0
  .\]
  The key is to use a $3\epsilon$ argument.
  Since $f(z) = S_N + E_N$, we have
  \[
    \left|\frac{f(z + h) - f(z)}{h} - g(z)\right|
    \le \left|\frac{S_N(z + h) - S_N(z)}{h} - S_N'\right|
    + |S_N' - g| + \left|\frac{E_N(z + h) - E_N(z)}{h}\right|
  .\]
  The first term is less than $\epsilon$ since
  $S_N$ is a polynomial, which are holomorphic.
  The second term is small since $g$ is a convergent
  power series and $g \to S_N'$. The final term is
  small since $E_N$ is an error term.
\end{proof}

\begin{remark}
  Really the miracle about complex analysis is
  that holomorphicity implies analyticity, and analytic
  functions are the ones that have all these nice
  properties (e.g. infinite differentiability, etc.).
\end{remark}
