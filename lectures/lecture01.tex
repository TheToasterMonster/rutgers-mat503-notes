\chapter{Introduction}
\section{Historical Motivation}
\marginpar{\small Lecture 1}
What is ``complex analysis''? The complex numbers are:
\[
  \C = \{x + iy \mid x, y \in \R\}
.\] 
``Analysis'' is a fancy way of saying ``calculus''.

So why do we want to study calculus over $\C$?
Why bother with $\C$ at all?

To solve quadratics? For example, the equation
\[x^2 + 1 = 0\]
yields the solutions $x = \pm i = \pm \sqrt{-1}$. No!
This is the same problem as asking where does the
function
\[y = x^2 + 1\]
cross the $x$-axis. But, of course, the graph of this
function doesn't intersect the $x$-axis at all! So there
is no reason to expect a solution for $x$ here.

Historically, mathematicians needed $i = \sqrt{-1}$
to solve cubic equations.
For example, consider the equation
\[y = x^3 + 12x - 15.\]
$y \to -\infty$ as $x \to -\infty$ and $y \to \infty$
as $x \to \infty$, so there must be a root somewhere by
the intermediate value theorem!

\subsection{Cardano's Story}
c.~1495, Paciolli in Italy writes a textbook about all
known mathematics at the time. Quadratics had already
been solved everywhere, but cubics were still a mystery
(Paciolli says ``cubics are as unsolvable as squaring
the circle'').

c.~1510, del Ferro figures out how to
solve the depressed cubic (no \\ quadratic term):
\[ax^3 + cx + d = 0\]

At the time, mathematicians were employed by the rich,
and to obtain such a position, one must win a
\textit{duel} against the current person holding the
position. Each contestant would give the other a set
of problems, and whoever solves more would win.

Thus, del Ferro doesn't tell anyone about his solution,
so he can use it as a secret weapon to win duels, if
necessary. Eventually, on his deathbed in the
1520s, he ends up telling his student Fior the secret.
Fior then
uses this to attack other mathematicians and win duels
all over the place. That is, until he attacks Tartaglia,
a renowned mathematician at the time.
\footnote{Tartaglia gave one of the first Latin
translations of Euclid's \textit{Elements}.}

Tartaglia sends Fior a set of regular problems, 
whereas Fior sends him 20 depressed cubics.
Tartaglia sees this and reasons that there must now
exist a solution to the depressed cubic, contrary to
common belief at the time (thus explaining Fior's
choice of problems). Knowing this, he rediscovers
the solution to the depressed cubic and proceeds to
win the duel. The public, seeing this, makes the same
conclusion that the depressed cubic has likely been
solved.

c.~1530, Cardano visits Tartaglia and asks him for the
solution, in the name of adding it to his textbook
(an update to Paciolli's) and promising to credit
Tartaglia. Tartaglia refuses, wanting to write his
own book.

However, after inviting Tartaglia to dinner and lots
of drinks, Cardano eventually convinces Tartaglia.
Tartaglia makes Cardano \\ solemnly swear to not reveal
the solution to the public, and he does so.

Later on, Ferrari becomes a student of Cardano and
eventually his collaborator. Ferrari eventually learns
the secrets, and together Ferrari and Cardano solves
all cubics (and all quartics too)! But they are unable
to publish their findings due to the oath.

They later go on a trip to Bologna, where they are
shown del Ferro's notes. There they find his original
solution to the depressed cubic, sitting in plain sight
for the past $30$ years (predating Tartaglia)!

Cardano proceeds to publish his book containing the
solution, the \textit{Ars Magna}, in 1545. Tartaglia
is not happy, and challenges Cardano and Ferrari to a
duel. Of course Tartaglia gets decimated as Ferrari
already knows how to solve even quartics. They almost
get into an actual duel, but Tartaglia manages to
escape before that can happen.


\begin{tcolorbox}[title=Note: Negative numbers, sharp corners, breakable, enhanced, parbox=false]
Interestingly, mathematicians cared about $i$ even
before they cared about negative numbers! How do we know
this? On top of considering cubics geometrically (with
actual cubes), Cardano considered the following cases:
\begin{gather*}
  x^3 + c = dx^2 \\
  x^3 = c + dx^2 \\
  x^3 + dx^2 = c.
\end{gather*}
It's evident that these 3 cases are all the same if
we take into account negative numbers, but he didn't see
this!
\end{tcolorbox}

\subsection{Solving the Cubic}
So, why were they forced to acknowledge $i$, if they
didn't even use negative numbers? Consider the general
cubic equation
\[ax^3 + bx^2 + cx + d = 0, \quad a \ne 0.\]
We can divide by $a$ (or equivalently let $a = 1$) to
make the equation \textit{monic}. Then we \textit{depress}
the cubic by making the change of variables
\[x = y - \frac{b}{3}.\]
So we have
\[\left(y - \frac{b}{3}\right)^3 + b\left(y - \frac{b}{3}\right)^2 + c\left(y - \frac{b}{3}\right) + d= 0.\]
Notice that the $y^2$ term in this expansion is
\[\binom{3}{1}y^2 \left(-\frac{b}{3}\right) + b\binom{2}{0}y^2 = -by^2 + by^2 = 0,\]
so we have eliminated the quadratic term.

Thus we need only consider cubics of the form
\[y^3 + Ay + B = 0.\]

\begin{tcolorbox}[title=Note: Quadratic equations, sharp corners, breakable, enhanced, parbox=false]
  How to solve  the quadratic $ax^2 + bx + c = 0$?

  First make it monic:
  \[x^2 + \frac{b}{a}x + \frac{c}{a} = 0.\]
  Depress it by letting
  \[
  x = y - \frac{b}{2a}
  .\] 
  Then we have
  \[
    \left(y - \frac{b}{2a}\right)^2 + \frac{b}{a}\left(y - \frac{b}{2a}\right) + \frac{c}{a} = 0
  ,\] 
  \[
    y^2 - \cancel{2 \frac{b}{2a} y} + \frac{b^2}{4a^2}
  + \cancel{\frac{b}{a} y} - \frac{b^2}{2a^2}
  + \frac{c}{a} = 0
  ,\]
  \[
    y^2 - \frac{1}{4} \frac{b^2}{a^2} + \frac{4ca}{4a^2}
    = 0
  ,\] 
  \[
  y^2 = \frac{b^2 - 4ac}{4a^2}
  .\] 
  From here, taking square roots and shifting by
  $\frac{b}{2a}$ again yields the usual quadratic formula.
\end{tcolorbox}

Depressing the quadratic makes it trivial, but the
same is not true for cubics! At first sight, the depressed
cubic is not any easier than the general cubic.
But the key insight is actually to perform some seemingly
unnecessary auxiliary calculations.
