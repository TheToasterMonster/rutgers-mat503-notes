\chapter{Differentiation}
\marginpar{\small Lecture 2}

Given a function $f : \C \to \C$, when is $f$
differentiable?

\section{Revisiting Multivariable Calculus}
When is a function $f : \R^2 \to \R$ differentiable?

Recall that calculus is \textit{linearization}. That is,
given something complicated, zoom in close enough and
it looks like a line.

\begin{tcolorbox}[title=Definition (Little-$o$)]
  We say $E = o(|h|)$ if
  \[\frac{E}{|h|} \to 0\]
  as $|h| \to 0$.
\end{tcolorbox}

\begin{tcolorbox}[title=Definition (Differentiability in $\R^2$)]
  $f : \R^2 \to \R^2$ is \textit{differentiable} at
  $z$ if
  \[f(z + h) = f(z) + Lh + o(|h|),\]
  where $L$ is a linear map.
\end{tcolorbox}

In other words, $f : \R^n \to \R^m$ is differentiable if
it is \textit{locally affine}. If
\[
f(x_1, \dots, x_m) = (f_1(\vec{x}), \dots, f_m(\vec{x}))
,\]
then
\[
f(z + h) = f(z) + Lh + o(|h|)
.\] 
Here $L$ is the \textit{total derivative}, given by
\[
  L = \left(\frac{\partial f_i}{\partial x_j}\right)_{ij}
.\]

\section{Holomorphic Functions}
We can identify a function $f(x, y) = (u, v)$ with
the complex valued function $f(x + iy) = u + iv$, where
we say that $u = \Re f$ and $v = \Im f$. Here,
\[
L =
\left(\begin{matrix}
  u_x & u_y \\
  v_x & v_y
\end{matrix}\right)
.\] 

Using the additional structure of the complex numbers
(instead of just 2D vectors), we can write
\[\frac{f(z + h) - f(z)}{h} = L + o(1).\]
Notice that dividing by $h$ makes no sense in $\R^n$, but
we can over $\C$.

\begin{tcolorbox}[title=Definition (Holomorphic)]
  $f : \C \to \C$ is \textit{holomorphic} at $z$ if
  \[
    \lim_{h \to 0} \frac{f(z + h) - f(z)}{h}
  \]
  exists. If it does, then we call this limit $f'(z)$.
\end{tcolorbox}

\begin{tcolorbox}[title=Definition (Continuity)]
  $f$ is \textit{continuous} if for all $\epsilon > 0$,
  there exists a $\delta > 0$ such that whenever
  $|h|  < \delta$, we have
  \[
  |f(z + h) - f(z)| < \epsilon
  .\] 
\end{tcolorbox}

How can we visualize these functions?
To graph a function $f : \R^2 \to \R^2$, we would need
$\R^4$ with 4 dimensions. This is hard. Instead,
we show a ``before'' and ``after'' image.

% insert of a picture of the above here

\subsection{Classic Example: Smooth but Not Analytic}
Let
\[
f(z) =
\begin{cases}
  e^{-1 / z^2}, & z \neq 0 \\
  0, & z = 0.
\end{cases}
\] 

Over $\R$, as $z \to 0$, $-\frac{1}{z^2} \to -\infty$
and $f \to 0$.  And as $|z| \to \infty$, $f \to e^0 = 1$. 
So $f$ is continuous. $f$ is also differentiable,
with
\[f'(z) = 2z^{-3}e^{-\frac{1}{z^2}}.\]
Notice that as $z \to 0$, $f' \to 0$ as well. What
about further derivatives?
\[
  f^{(n)}(z) = P_n(z^{-1})e^{-\frac{1}{z^2}}
,\] 
so $f^{(n)} \to 0$ also as $z \to 0$. So
$f \in C^\infty(\R)$.
\footnote{Meaning $f$ is \textit{smooth}, or infinitely
differentiable.}

Now take
\[
g(z) =
\begin{cases}
  e^{-1 / z^2}, & z > 0 \\
  0, & z \le 0.
\end{cases}
\]
This is an example of a \textit{smooth bump function}.
\footnote{These are used in real analysis, particularly
for \textit{partitions of unity}.}
All derivatives of $g$ are $0$ for $z > 0$ as before,
and they are also $0$ for $z \le 0$ by definition. So
$g$ is also smooth, with $f \equiv g$ on $z \ge 0$ and
$f \ne g$ on $z < 0$.

Notice that $f$ and $g$ both extend
$\left.e^{-1 / z^2}\right\rvert_{\R^+}$ to all of $\R$,
yet $f \ne g$. We will see later that this cannot
happen for complex-analytic functions.
\footnote{This is the fact that \textit{analytic continuations} are \textit{unique}.}

\begin{tcolorbox}[title=Definition (Analytic)]
  A function $f$ is \textit{analytic} at $z$ if there
  exists an $\epsilon > 0$ and $a_0, a_1, a_2, \dots$
  such that for all $|h| < \epsilon$, the series
  \[
  a_0 + a_1 h + a_2 h^2 + \dots
  = \sum_{i = 0}^{\infty} a_i h^i
  \] 
  converges absolutely and is equal to $f(z + h)$.
\end{tcolorbox}

In simpler words, $f$ is analytic at $z$ if it has a
locally convergent Taylor series expansion around $z$.

Now, is
\[
f(z) =
\begin{cases}
  e^{-1 / z^2}, & z \neq 0 \\
  0, & z = 0
\end{cases}
\] 
analytic at $0$? All the Taylor coefficients
are zero:
\[a_0 = a_1 = a_2 = \dots = 0.\]
Thus the series expansion of $f$ converges to $0$, but
there is no $\epsilon$-neighborhood of $0$ where
$f \equiv 0$.
So $f$ is \textit{not} analytic. This also cannot
happen for complex-valued functions.
\footnote{This is why the terms \textit{holomorphic} and
\textit{analytic} are often used interchangeably in
  complex analysis.}

What about $f(z) = e^{-1 / z^2}$ for $z \in \C$?
Now, $f$ isn't even continuous! If $z \to 0$
from $\R$, $f \to 0$. But if $z = iy$ with $y \to 0$,
\[f(z) = e^{-\frac{1}{(iy)^2}} = e^{{1}/{y^2}},\]
which approaches $\infty$. The two paths give conflicting
limits!
\footnote{In fact, you can actually make $f$ converge to
  any value by picking some path. This is because $f$
  has an \textit{essential singularity} at $0$.}

\section{The Cauchy-Riemann Equations}
